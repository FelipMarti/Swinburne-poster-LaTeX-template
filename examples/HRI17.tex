%%
%% This is file `tikzposter-template.tex',
%% generated with the docstrip utility.
%%
%% The original source files were:
%%
%% tikzposter.dtx  (with options: `tikzposter-template.tex')
%% 
%% This is a generated file.
%% 
%% Copyright (C) 2014 by Pascal Richter, Elena Botoeva, Richard Barnard, and Dirk Surmann
%% 
%% This file may be distributed and/or modified under the
%% conditions of the LaTeX Project Public License, either
%% version 2.0 of this license or (at your option) any later
%% version. The latest version of this license is in:
%% 
%% http://www.latex-project.org/lppl.txt
%% 
%% and version 2.0 or later is part of all distributions of
%% LaTeX version 2013/12/01 or later.
%% 
%% 
%% 
%% Copyright (C) 2016 by Felip Marti Carrillo - fmarti@swin.edu.au
%%
%% This Swinburne version of the poster template may be distributed  
%% and/or modified under the conditions described above 
%% 





\documentclass[a0paper, portrait]{tikzposter}
%%	Options for format can be included here
%%	a0paper, a1paper, a2paper
%%	portrait, landscape
%%	Check the documentation for more options

 % Title, Author, Institute
\title{\parbox{\linewidth}{\centering In-Situ Design and Development of a Socially Assistive Robot for Paediatric Rehabilitation}}
\author{F. Mart\'{i} Carrillo$^1$$^,$$^2$*, 
J. Butchart$^3$$^,$$^4$, 
S. Knight$^4$$^,$$^3$, 
A. Scheinberg$^3$$^,$$^4$, 
L. Wise$^1$, 
L. Sterling$^1$ 
and 
C. McCarthy$^1$
\\
$*$\url{fmarti@swin.edu.au}
}
\institute{
    $^1$Swinburne University of Technology, Melbourne, Australia \\
    $^2$Data61 -- CSIRO, Melbourne, Australia \\
    $^3$Royal Children's Hospital, Melbourne, Australia \\
    $^4$Murdoch Childrens Research Institute, Melbourne, Australia \\
}
\titlegraphic{
%\includegraphics[width=10cm]{figures/SwinLogoH}
}

 %Choose Layout
\usetheme{Swinburne}


% REDEFINING TITLE
\makeatletter
\renewcommand\TP@maketitle{%
      \tikz[remember picture,overlay]\node[scale=0.75,anchor=east,xshift=0.42\linewidth,yshift=11cm,inner sep=0pt] {%
        \includegraphics[trim=240 60 250 60,clip,width=11cm]{figures/royal-childrens-logo}
    };
   \centering
   \begin{minipage}[b]{0.8\linewidth}
        \centering
        \color{titlefgcolor}
        {\bfseries \Huge \@title \par}
        \vspace*{1em}
        {\Large \sl \@author \par}
        %\vspace*{1em}
        {\large \@institute}
    \end{minipage}%
      \tikz[remember picture,overlay]\node[scale=0.8,anchor=east,xshift=0.55\linewidth,yshift=8cm,inner sep=0pt] {%
        \includegraphics[width=10cm]{figures/SwinLogoV}
    %  \tikz[remember picture,overlay]\node[scale=0.8,anchor=east,xshift=0.5\linewidth,yshift=3cm,inner sep=0pt] {%
    %    \includegraphics[width=18cm]{figures/SwinLogoH}
   %    \@titlegraphic
    };
}
\makeatother
% END REDEFINING TITLE


\begin{document}

 % Title block with title, author, logo, etc.
\maketitle


 \begin{columns}

 % FIRST column
 \column{0.38}% Width set relative to text width
 \block{Introduction}{
    Socially Assistive Robots (SAR) show great potential for boosting therapeutic outcomes in children undergoing 
    intensive rehabilitation. 
    In partnership with a busy paediatric rehabilitation clinic, we are developing and evaluating software to adapt 
    NAO as a therapeutic aid for paediatric rehabilitation.
    Unlike previous work, we are focussed specifically on the needs of clinical deployment. In particular, NAO leads 
    therapy sessions for children with physical disabilities, such as cerebral palsy, 
    undergoing intensive rehabilitation. We aim to
    increase exercise compliance and maintain emotional wellbeing, particularly when therapists are not in attendance.
 }


 % SECOND column
 \column{0.62}% Width set relative to text width
 \block{Design Approach}
 {
     
    \begin{description}  
    \item [Phase 1 (Exploration):] 
    Has two key objectives: determination of SAR roles and requirements through rapid prototyping and 
    mock-ups (via \emph{Wizard-of-Oz} control); and establishing the legitimacy and acceptance of the technology with stakeholders.
    Activities are conducted predominantly on-site, over the course of frequent unstructured visits.
    Data is gathered through investigator observation and unstructured consultation with patients/parents, therapists and doctors.
    \item [Phase 2 (Formative Evaluation and Development):]
    Focusses on deployment and iterative development of the SAR.
    A stand-alone minimum viable SAR prototype (based on Phase 1 findings) is developed and deployed in patient sessions.
    Robot performance data is gathered during sessions, and patient/therapist/parent perceptions of trust, usefulness and therapeutic benefits are gathered  
    via semi-structured interviews at the completion of each session.
    \end{description}


 }
 \end{columns}

 \begin{columns}
 % FIRST column
 \column{0.1}% Width set relative to text width
 \column{0.8}% Width set relative to text width
 \block{}{
    \begin{tikzfigure}%[Project Phases]
        \includegraphics[trim=40 365 0 65,clip,width=0.75\columnwidth]{figures/phases}  
    \end{tikzfigure} 
 }
 \note[targetoffsetx=-44cm, targetoffsety=0cm, angle=0, rotate=10, width=11cm]
 {Over 30 patients engaged with NAO during Phase 1}
 \note[targetoffsetx=28cm, targetoffsety=1cm, angle=0, rotate=-3, width=10cm]
 {NAO has led 14 rehab sessions with 9 patients during Phase 2}

 \end{columns}


%\block{Rehabilitation Exercises from a Laying Down Position}{
%
%\begin{tikzfigure}%[Bridge]
%    \includegraphics[width=0.18\columnwidth]{figures/Bridge}
%    \includegraphics[width=0.18\columnwidth]{figures/HipAbdLaying}
%    \includegraphics[width=0.18\columnwidth]{figures/KneeExtensionSideLying}
%    \includegraphics[width=0.18\columnwidth]{figures/HipKneeFlexionEasy}
%    \includegraphics[width=0.18\columnwidth]{figures/QuadsOverRoll}
%\end{tikzfigure} 
%
%\begin{tabular}{p{0.005\columnwidth} p{0.17\columnwidth} p{0.005\columnwidth} p{0.17\columnwidth} p{0.005\columnwidth} p{0.17\columnwidth}
%                p{0.005\columnwidth} p{0.17\columnwidth} p{0.005\columnwidth} p{0.17\columnwidth}
%}
%& \centering \textbf{Bridge} &
%& \centering \textbf{Hip Abduction Laying} &
%& \centering \textbf{Knee Extension on Side} &
%& \centering \textbf{Hip Knee Flexion Easy} &
%& \centering \textbf{Quads over Roll} 
%\end{tabular}
%
%\begin{tikzfigure}%[Bridge]
%    \includegraphics[width=0.18\columnwidth]{figures/SingleBridge}
%    \includegraphics[width=0.18\columnwidth]{figures/HipAbdSideLie}
%    \includegraphics[width=0.18\columnwidth]{figures/LegRaises}
%    \includegraphics[width=0.18\columnwidth]{figures/HipKneeFlexionHard}
%    \includegraphics[width=0.18\columnwidth]{figures/StaticQuads}
%\end{tikzfigure} 
%
%\begin{tabular}{p{0.005\columnwidth} p{0.17\columnwidth} p{0.005\columnwidth} p{0.17\columnwidth} p{0.005\columnwidth} p{0.17\columnwidth}
%                p{0.005\columnwidth} p{0.17\columnwidth} p{0.005\columnwidth} p{0.17\columnwidth}
%}
%& \centering \textbf{Single Bridge} &
%& \centering \textbf{Hip Abduction On Side} &
%& \centering \textbf{Leg Raises} &
%& \centering \textbf{Hip Knee Flexion Easy} &
%& \centering \textbf{Static Quads} 
%\end{tabular}
%
%
%}
% \note[targetoffsetx=25cm, targetoffsety=16cm, angle=0, rotate=-5, width=10cm]
% {}

\begin{columns}
 % FIRST column
 \column{0.3}% Width set relative to text width
 \block{Roles (from Phase 1)}
 {
    \begin{description}
    \item[Demonstrator:]  NAO demonstrates exercises
    at the beginning of each set, and provides verbal instructions.

    \item[Motivator:]  NAO provides verbal encouragement at the beginning, 
    during, and at the end of each prescribed exercise.  
    Entertainment through music, dancing and joke telling are also 
    offered upon completion of exercise sets.  

    \item[Companion:] NAO is a co-participant during the session,   
    joining in and providing empathetic statements to acknowledge the child's progress.
    \end{description}
 }

 \column{0.7}
 \block{Derived Requirements (from Phase 1 and 2)}
 {
  \begin{minipage}{0.31\columnwidth}

    \begin{description}
    \item [Configurability:] 
    Therapists must be able to pre-load rehabilitation exercises, number of repetitions, etc.
    
    \item [Stability:]
    To minimise failure, demonstration exercises must utilise joint poses 
    and movements that remain within conservatively defined operating limits. 
    
    \item [Adaptability:]
    To ensure the therapeutic assistance is aligned with the patient's presenting needs,
    the SAR should adapt to patient mood and progress, allowing 
    in-session adjustment of activity settings (eg, repetitions, speed and sequence order).

    \item [Integration:] 
    The SAR must be easily set up, portable and operable by carers without specialised training. 
    
    \end{description}
  \end{minipage}
  \begin{minipage}{0.01\columnwidth}
  \textcolor{white}{.}
  \end{minipage}
  \begin{minipage}{0.31\columnwidth}
    \begin{description}
    \item [Interaction:]
    Basic interaction with the SAR should be supported for both carer and patient throughout the session.
    This will support Adaptability, Responsiveness, and maintain patient engagement.
    
    \item [Responsiveness:] 
    The SAR should %broadly 
    recognise the patient's mood and progress, and respond appropriately. %(therapist-configured).
    
    \item [Stand-alone:] 
    %The SAR should be operable without on-site technical support, \emph{Wizard-of-Oz} control, 
    %or additional hardware.  
    The SAR should be operable without technicians, \emph{Wizard-of-Oz} or additional hardware.  
    
    \item [Robustness and Endurance:]
    The system should operate continuously. %without engineer intervention. 
    Unforeseen interruptions such as falls, or unintended user interactions 
    should be restorable. 
    \end{description}
  \end{minipage}
 }
\end{columns}

\begin{columns}
 % FIRST column
 \column{1}% Width set relative to text width
 \block{Socially Assistive Robot Roles in Action}
 {
    \begin{tikzfigure}%[CAPTION HERE]
        \includegraphics[trim=10 0 205 5,clip,width=0.31\columnwidth]{figures/DemoBubble}
        \includegraphics[trim=0 0 200 0,clip,width=0.31\columnwidth]{figures/SitStands}
        \includegraphics[trim=0 0 0 0,clip,width=0.31\columnwidth]{figures/ToyRelayBubble}
    \end{tikzfigure} 
%\begin{tabular}{p{0.005\columnwidth} p{0.17\columnwidth} p{0.005\columnwidth} p{0.17\columnwidth} p{0.005\columnwidth} p{0.17\columnwidth}
%                p{0.005\columnwidth} p{0.17\columnwidth} p{0.005\columnwidth} p{0.17\columnwidth}
%}
\begin{tabular}{p{0.005\columnwidth} p{0.29\columnwidth} p{0.01\columnwidth} p{0.29\columnwidth} p{0.01\columnwidth} p{0.29\columnwidth}
                p{0.01\columnwidth} 
}
& NAO \textbf{demonstrates} a ``bridge'' to the patient &
& NAO \textbf{accompanies} the patient during the Sit-to-Stand exercise & 
& NAO \textbf{motivates} the patient during the Toy Relay game
\end{tabular}
 }

\end{columns}


\begin{columns}
 \column{0.1}
 \column{0.5}
 \block{Design Process Evaluation}
 {
    \begin{itemize}
        \item Frequent (weekly) hospital visits and \textbf{in-situ development} have been 
        \textbf{key} to the \textbf{high volume of patient interactions} driving the SAR development
        \item The \textbf{direct inclusion of therapists} in the co-design of the system 
        has been a \textbf{key} component in \textbf{building trust and ownership} of the resulting SAR prototype
        \item Requires \textbf{large time investment} of a small development team 
        \item Promotes design transparency, but \textbf{exposes system deficits} to the end users
    \end{itemize}
 }
 % SECOND column
 \column{0.3}
 \block{Acknowledgements}
 {
    \begin{itemize}
        \item Project Funding: Traffic Accident Commission (TAC) and Data61 -- CSIRO
        \item Thanks to all the patients, parents and therapists who have engaged with NAO
    \end{itemize}
 }
\end{columns}


\end{document}



\endinput
%%
%% End of file `tikzposter-template.tex'.
