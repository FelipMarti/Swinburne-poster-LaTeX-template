%%
%% This is file `tikzposter-template.tex',
%% generated with the docstrip utility.
%%
%% The original source files were:
%%
%% tikzposter.dtx  (with options: `tikzposter-template.tex')
%% 
%% This is a generated file.
%% 
%% Copyright (C) 2014 by Pascal Richter, Elena Botoeva, Richard Barnard, and Dirk Surmann
%% 
%% This file may be distributed and/or modified under the
%% conditions of the LaTeX Project Public License, either
%% version 2.0 of this license or (at your option) any later
%% version. The latest version of this license is in:
%% 
%% http://www.latex-project.org/lppl.txt
%% 
%% and version 2.0 or later is part of all distributions of
%% LaTeX version 2013/12/01 or later.
%% 
%% 
%% 
%% Copyright (C) 2016 by Felip Marti Carrillo - fmarti@swin.edu.au
%%
%% This Swinburne version of the poster template may be distributed  
%% and/or modified under the conditions described above 
%% 





\documentclass[a0paper, portrait]{tikzposter}
%%	Options for format can be included here
%%	a0paper, a1paper, a2paper
%%	portrait, landscape
%%	Check the documentation for more options

 % Title, Author, Institute
\title{\parbox{\linewidth}{\centering`Help me help you': A Human-Assisted Social Robot in Paediatric Rehabilitation}}
\author{F. Mart\'{i} Carrillo$^1$$^,$$^2$*, 
J. Butchart$^3$$^,$$^4$, 
S. Knight$^4$$^,$$^3$, 
A. Scheinberg$^3$$^,$$^4$, 
L. Wise$^1$, 
L. Sterling$^1$ 
and 
C. McCarthy$^1$
\\
$*$\url{fmarti@swin.edu.au}
}
\institute{
    $^1$Swinburne University of Technology, Melbourne, Australia \\
    $^2$Data61 -- CSIRO, Melbourne, Australia \\
    $^3$Royal Children's Hospital, Melbourne, Australia \\
    $^4$Murdoch Childrens Research Institute, Melbourne, Australia \\
}
\titlegraphic{
%\includegraphics[width=10cm]{figures/SwinLogoH}
}

 %Choose Layout
\usetheme{Swinburne}


% REDEFINING TITLE
\makeatletter
\renewcommand\TP@maketitle{%
      \tikz[remember picture,overlay]\node[scale=0.75,anchor=east,xshift=0.46\linewidth,yshift=4cm,inner sep=0pt] {%
        \includegraphics[width=22cm]{figures/data61OnBlack}
    };
   \centering
   \begin{minipage}[b]{0.8\linewidth}
        \centering
        \color{titlefgcolor}
        {\bfseries \Huge \@title \par}
        \vspace*{1em}
        {\Large \sl \@author \par}
        %\vspace*{1em}
        {\large \@institute}
    \end{minipage}%
      \tikz[remember picture,overlay]\node[scale=0.8,anchor=east,xshift=0.55\linewidth,yshift=8cm,inner sep=0pt] {%
        \includegraphics[width=10cm]{figures/SwinLogoV}
    %  \tikz[remember picture,overlay]\node[scale=0.8,anchor=east,xshift=0.5\linewidth,yshift=3cm,inner sep=0pt] {%
    %    \includegraphics[width=18cm]{figures/SwinLogoH}
   %    \@titlegraphic
    };
}
\makeatother
% END REDEFINING TITLE


\begin{document}

 % Title block with title, author, logo, etc.
\maketitle


 \begin{columns}

 % FIRST column
 \column{0.45}% Width set relative to text width
 \block{Abstract}{
    Socially assistive robots show great potential for boosting therapeutic outcomes in children undergoing intensive
    rehabilitation. However, the introduction of an additional interactive presence also imposes new demands on the
    therapist. 

    In this preliminary study we explore the time costs and issues associated with the inclusion of a 
    semi-autonomous assistive robot in paediatric rehabilitation sessions.
 }

 \block{The study}{
    \begin{itemize}
        \item \textbf{Five} normal patients' rehabilitation \textbf{sessions}
        \item \textbf{Three} different \textbf{patients}: 2 girls (9 and 11 years old) and 1 boy (6 years old)
        \item \textbf{Each session varied} with patients' needs. For instance, in session 3 only  
        activities requiring \emph{Helping to Keeping Pace} assistance were used. 
    \end{itemize}
}

 % SECOND column
 \column{0.55}% Width set relative to text width
 \block{Human-Supported Capabilities}
 {
     
    \begin{description}  
    \item [Positioning the robot:] 
        During a typical rehabilitation session, NAO will perform activities in a range of locations. The robot starts each
        session placed on the floor, and must be positioned close to the patient on an adjacent bed or raised table in the line
        of sight of the patient.
        NAO requests this assistance by asking the attending adult to position it appropriately.
    \item [Posture:]
        NAO cannot archive certain postures without human assistance. When needed,
        at the beginning of the exercise NAO changes posture and asks to be rolled onto its side.
    \item [Placing auxiliary aids:] \textcolor{white}{Two different aids}
        \begin{itemize}  
        \item \textbf{Rolled Towel:} \emph{Quads over Roll} and \emph{Static Quads} 
            are examples of exercises which require a soft, cylindrically rolled towel
            to be placed under both the patient and robot's knees.
            Both require human assistance and so NAO requests this explicitly.
        \item \textbf{Seat:}
        Exercises such as \emph{Sit to Stand} and \emph{Quads Strengthening}
        are done from a sitting position. Therefore, the robot requires positioning of a seat behind its legs.
        \end{itemize}
        
    \item [Helping to keep pace:] 
        Fatigue, distraction and frustration can all affect the speed and timing of the rehabilitation session. 
        It is crucial that NAO keeps pace and allows the patient to rest. % and regain focus between different exercises.
    \end{description}

 }
 \end{columns}

\block{Assisting the robot}{

\begin{tikzfigure}%[CAPTION HERE]
        \includegraphics[trim=500 0 500 0,clip,width=0.18\columnwidth]{figures/help1a}
        \includegraphics[trim=300 0 700 0,clip,width=0.18\columnwidth]{figures/help4a}
        \includegraphics[trim=500 0 500 0,clip,width=0.18\columnwidth]{figures/help2a}
        \includegraphics[trim=1000 0 0 0,clip,width=0.18\columnwidth]{figures/help3a}
        \includegraphics[trim=0 0 1000 0,clip,width=0.18\columnwidth]{figures/help5a}
        \includegraphics[trim=300 0 700 0,clip,width=0.18\columnwidth]{figures/help1b}
        \includegraphics[trim=500 0 500 0,clip,width=0.18\columnwidth]{figures/help4b}
        \includegraphics[trim=500 0 500 0,clip,width=0.18\columnwidth]{figures/help2b}
        \includegraphics[trim=500 0 500 0,clip,width=0.18\columnwidth]{figures/help3b}
        \includegraphics[trim=0 0 1000 0,clip,width=0.18\columnwidth]{figures/help5b}
\end{tikzfigure} 


\begin{tabular}{p{0.005\columnwidth} p{0.17\columnwidth} p{0.005\columnwidth} p{0.17\columnwidth} p{0.005\columnwidth} p{0.17\columnwidth}
                p{0.005\columnwidth} p{0.17\columnwidth} p{0.005\columnwidth} p{0.17\columnwidth}
}
& \textbf{Positioning the robot:} Robot laying down by itself on the floor (top).
Robot being placed by the physiotherapist on a table (bottom). &
& \textbf{Posture:} Initial position of the robot when asking to be rolled (top).
Physiotherapist rolling the robot onto its right side (bottom). &
& \textbf{Placing a rolled towel:} Initial position of the robot when asking for a towel (top).
Physiotherapist putting a towel under robot's knee (bottom). &
& \textbf{Placing a seat:} Physiotherapist putting a seat behind the robot (top). 
Robot sitting, ready to start sitting exercises (bottom). &
& \textbf{Helping to keep pace:} Robot's brain LEDs blinking, waiting to be tapped (top).
Patient tapping robot's head (bottom).
\end{tabular}

}
 \note[targetoffsetx=25cm, targetoffsety=16cm, angle=0, rotate=-5, width=10cm]
 {Keeping Pace actions appeared to increase patient engagement!}

\begin{columns}
 % FIRST column
 \column{0.3333}% Width set relative to text width
 \block{Time required to help the robot in a 30 minute session}
 {
    \begin{tikzfigure}%[CAPTION HERE]
        \includegraphics[width=0.29\columnwidth]{figures/T}
    \end{tikzfigure} 
 }
 \note[targetoffsetx=-16.5cm, targetoffsety=8cm, angle=0, rotate=10, width=10cm]
 {Less than 2 minutes in a 30 minute session!}

 % SECOND column
 \column{0.3333}% Width set relative to text width
 \block{Results}
 {
    \begin{itemize}
        \item \textbf{Positioning the robot took the longest time}, influenced by the proximity of the physiotherapist or the room layout.
        \item Assisting the robot to \textbf{keep pace} requires less time, but \textbf{occurs more often}. 
                This kind of assistance scales roughly with the number of activities.
        \item \textbf{Therapists expressed no concern with this time cost} in the sessions, and the assistance requests are in an acceptable upper limit.
        \item  \textbf{Keeping Pace actions} appeared to complement the general desire of patients to interact with the robot.
                Therapist feedback indicated that allowing \textbf{patients} to assist NAO \textbf{appeared to increase their activity and engagement} during the session.
    \end{itemize}
 }

 % THIRD column
 \column{0.3333}
 \block{Number of times help is needed in a 30 minute session}
 {
    \begin{tikzfigure}%[CAPTION HERE]
        \includegraphics[width=0.29\columnwidth]{figures/O}
    \end{tikzfigure} 
 }
 \note[targetoffsetx=1cm, targetoffsety=7.5cm, angle=0, rotate=-10, width=9.5cm]
 {Therapists expressed no concerns about the help requirements!}
\end{columns}

\begin{columns}
 % FIRST column
 \column{0.1}
 \column{0.5}
 \block{Conclusions}
 {
    \begin{itemize}
        \item Physiotherapists dealing with the robot were not over-burdened by the system's help requests. Indeed, it helped to engage patients.
        \item We are currently exploring the development of patient monitoring capabilities, a clinicians' interface and other refinements in preparation for
        a planned clinical trial in 2017.
    \end{itemize}
 }
 % SECOND column
 \column{0.3}
 \block{Acknowledgements}
 {
    \begin{itemize}
        \item Project Funding: Traffic Accident Commission (TAC) and Data61 -- CSIRO
        \item Thanks to all the patients, parents and therapists who have engaged with NAO
    \end{itemize}
 }
\end{columns}


\end{document}



\endinput
%%
%% End of file `tikzposter-template.tex'.
